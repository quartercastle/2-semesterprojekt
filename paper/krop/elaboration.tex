\documentclass[../main.tex]{subfiles}

\begin{document}

\section{Elaboration}
I dette afsnit vil der blive redegjort for arbejdet i elaborationsfasen. Helt overordnet handler elaborationsfasen om at lave en delvis funktionel version af systemet, der kan benyttes som grundlag for udviklingen af resten af systemet i konstruktionsfasen. Der er især fokus på disse centrale arbejdsprocesser:

\begin{itemize}
  \item Krav udledning: Handler om at forbedre systemet omfang og krav. Dette gøres ved at udarbejde en overordnet kravspecifikation og en detaljeret kravspecifikation.
  \item Analyse: Handler om at finde ud af hvad der skal bygges. Dette gøres ved hjælp af analysemodellen.
  \item Design: Handler om at udtænke den bedste arkitektur til systemet, og finde frem til de konkrete implementeringsdetaljer. Dette gøres ved hjælp af designmodellen.
  \item Implementering: Handler om at udvikle systemet. Dette gøres ved at udarbejde kode i et objektorienteret sprog.
  \item Test: Handler om at teste det byggede system.
\end{itemize} \cite{arlow2005uml}

Da der i dette projekt kun arbejdes i inceptions- og elaborationsfasen vil der være et større fokus på at kode i elaborationsfasen, end der normalt vil være i et projekt indeholdende alle fire UP faser. 

\subfile{krop/elaboration/overordnet_krav}

\subfile{krop/elaboration/detaljeret_krav}

\subfile{krop/elaboration/analyse}

\subfile{krop/elaboration/design}

\subfile{krop/elaboration/implementering}

\subfile{krop/elaboration/login}

\subfile{krop/elaboration/sags_aabning}

\subfile{krop/elaboration/persistens_implementering}

\subfile{krop/elaboration/test}

\subsection{Konklusion}
Som resultat af elaborationsfasen er der blevet udarbejdet en god implementering baseret på vores analyse og design, lavet ud fra voksenudredningsmetoden. Kombinationen af UP og Scrum har været udfordrende, men efterhånden som der kom flere erfaringer undervejs, endte det med at kombinationen fungerede til sidst. Scrum har været en stor del af arbejdet med elaborationsfasen. Det har hjulpet til ved at strukturere og skabe et overblik over igangværende og færdige brugsmønstre i form af sprint backloggen og burndown diagrammerne. Derudover har estimering af opgaver været svært, men samtidig givet nogle gode erfaringer, da det har været en god indikator for, om en opgave er færdiggjort, og om der har været for mange abstraheringer.

\end{document}