\begin{document}

\section{Brug af UP og Scrumbuts}


KLADE KLADE KLADE KLADE KLADE KLADE KLADEKLADE KLADE KLADE KLADEKLADEKLADEKLADEKLADEKLADE
Unified Process (UP) er en iterativ og inkrementel proces. UP bygger på konceptet, at det er nemmere at løse mange små problemer, end at skulle løse ét enkelt stort problem. UP opdeles i fire faser: inception, elaboration, construction og transition.
I inceptionsfasen blev projektet overvejet, og vi fik dannet et overblik over hvilke krav der skulle stilles til projektet. Målet for inceptionsfasen var at få projektet gjort ordentligt klar til at kunne gå effektivt til værks.
I elaborationsfasen skal der udvikles centrale komponenter af systemet, samt opnås en større forståelse for kravene til systemet, og systemets arkitektur. Målet for elaborationsfasen er at få lavet et færdigt produkt der i sig selv overholder arkitekturen, men som stadig kan videreudvikles i de kommende faser. Det er en af de mest kritiske faser i forløbet, da de resterende faser i projektet afhænger af resultatet herfra.

Scrum er en agil metode, baseret på ”sprints”. Sprints er korte udviklingscyklusser på højst en måned hver, hvor der udvikles et potentielt ”færdigt” system, klar til at blive udgivet. Tidsplanen for Sprints er fastsat, og ikke kan ændres. I Scrum arbejdes der i et Scrum Team. Det består af selve udviklingsholdet, som står for selve arbejdet på hver increment, en Scrum Master, som er en leder for holdet, der styrer dem i den rigtige retning, og en Product Owner, som er en direkte kontakt til kunden, der ved hvad produktet skal indeholde og hvilken retning det skal føres i.
Udover selve Sprinten findes der i Scrum 4 hoved events:
Sprint planning, hvor hele Scrum holdet samlet planlægger arbejdet for den kommende Sprint. Dette bliver afholdt én gang per Sprint, som det første.
Daily Scrum, hvor der på et hurtigt 15-minutters møde bliver aftalt arbejdet for de næste 24 timer. Som det fremgår af navnet bliver Daily Scrum afholdt dagligt i Sprints.
Sprint review, hvor den seneste Sprint evalueres og produkt backlog opdateres. Der bliver reviewet i slutningen af hver Sprint.
Sprint retrospective, hvor der bliver diskuteret og taget i betragtning hvad der gik godt og hvad der kan forbedres til næste Sprint. Sprint retrospective bliver holdt mellem afslutning af et Sprint og starten af et Sprint.
Sprints i Scrum ligger ryg mod ryg, så når ét sprint afsluttes, starter det næste så hurtigt som muligt.
Vi vil i elaborationsfasen benytte os af en kombination af de to metoder. Det sker ved at vi gennem en blanding af de to metoder vil bruge de elementer fra hver, som vi har vurderet passer bedst til netop denne fase. Det overordnede projekt er allerede opdelt i de fire faser fra Unified Process. Dette betyder at vi fra elaborationsfasens start allerede har et godt grundlag for at gå i gang, i form af dette inceptionsdokument og de overvejelser vi har gjort os.
Som supplement til dette vil vi benytte os af en mindre fuldbyrdet version af Scrum. Da vi i form af projektets opbygning ikke har mulighed for at arbejde sammen med en product owner, vil det i stedet være vores egen opgave at kunne sætte os i kundens sted. Herudover vil det højst sandsynligt ikke være muligt at udnævne en ”ægte” Scrum master, da vi nok får brug for alle gruppens medlemmer til at bidrage i udviklingen af systemet. Vi vil i projektet gøre brug af Sprints, og de kommer højst sandsynlig til at blive afviklet ugentligt, hvor vi bestemmer sprint planning tirsdag, holder sprint review fredag, holder et daily scrum alle tre dage og laver sprint retrospective et sted mellem fredag og tirsdag.
