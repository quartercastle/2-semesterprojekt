\documentclass[../../main.tex]{subfiles}

\begin{document}

\subsubsection{Login}

I dette afsnit vil implementeringen af login brugsmønsteret blive beskrevet og forklaret.

	På kode eksempel \ref{AuthManager.verify()} ses implementeringen af metoden \code{verify}, som har ansvaret for at finde en \code{User} i databasen, ud fra et matchende username. Hvis der bliver fundet en user med matchende \code{username} i databasen, vil næste skridt være at verificere, at det givne password matcher med det, der blev fundet i databasen. 
Hvis ingen bruger blev fundet med matchende username, vil systemet afbryde og returnere værdien \code{null} for at indikere, at authentificeringen fejlede. Ydermere vil metoden også afbryde, hvis det givne password ikke matcher med den fundne brugers password. Som det ses i kode eksemplet herunder, er instansen af den fundne \code{user} information expert, da den har ansvaret for at verificere at passwordet matcher med sit eget. Dette er valgt da det gør selve \code{AuthManager} implementeringen overskuelig, og det er nemt hurtigt at danne sig et overblik over, hvornår en bruger bliver authentificeret og hvornår der afbrydes.
Når en bruger authentificeres returneres der en instans af brugeren, der har fået adgang. Dette muliggør, at man kan oprette en \code{CaseWorker} med rettighederne brugeren har til rådighed. \\

\begin{lstlisting}[caption=AuthManager.verify(), captionpos=b, label=AuthManager.verify()]
  public static IUser verify(String username, String password) {
    User user = (User) Mapper.map(
    	DomainFacade.getInstance().getData().findUser(username), 
        false
    );

    if (user == null) {
      return null;
    }

    if (!user.verify(password)) {
      return null;
    }

    return user;
  }
\end{lstlisting}



\end{document}
