\documentclass[../main.tex]{subfiles}

\begin{document}
\section*{Abstract}
\addcontentsline{toc}{section}{Abstract}
This report was written as part of documenting the semester assignment of the second semester for Software Engineering students at SDU.\\
The assignment was build around a case handling system, designed for handicapped adults, both physically and mentally, called Sensum Udred. Sensum Udred was being considered for development by the company EG Team Online A/S, and thus they needed to do preemptive research before making the decision to go ahead with the project. The assignment consisted of examining the market EG wished to enter, and analyze it, with the goal of determining whether or not the project was worth taking on or not. Thereafter, a prototype of some core functionality needed to be developed. The development was done using Scrum, with modifications,	 for the overall project management, and Unified Process for the analysis, design and development. The prototype was made with the goal to prove, that it was possible to make the case handling process more efficient, while maintaining the practices described in the Adult Elucidation Method (Voksenudredningsmetoden), and not compromising the safety of the citizens data. \\
The development was done with a focus on case-opening, and followed the methods described in Unified Process, from the inception phase, to the elaboration phase, thus leaving out the final two phases: Construction and Transition. 

The report and it's appendix contains documentation of all parts of the process, including explanations of the way the system has been implemented, and the considerations behind these decisions. Furthermore it contains documentation of the database used for storing the data, implemented in such a way that it doesn't break the General Data Protection Regulation.

The report was handed in on the 31st of may, 2018, along with the source code for the prototype of the case-opening functionality of Sensum Udred.




\newpage
\end{document}
